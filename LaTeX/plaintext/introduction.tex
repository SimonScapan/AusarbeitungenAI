% !TEX root =  master.tex
\chapter{Einleitung}

\nocite{*}

Diese wissenschaftliche Ausarbeitung beschäftigt sich mit Themen aus dem Bereich \ac{AI}, welche im Zusammenhang mit gleichnamiger Vorlesung behandelt wurden. Dabei wird auf vier Themengebiete eingegangen:

\begin{enumerate}
	\item Historische Entwicklung von \ac{AI}
	\item Gesellschaftliche Auswirkungen von \ac{AI}
	\item Evolutionäre und Genetische Algorithmen zur Bewältigung komplexer Aufgaben
	\item Generative Verfahren der \ac{AI}
\end{enumerate}

Für jedes Themengebiet wird in dem jeweiligen Folgekapitel ein bestimmter Bereich oder ein Thema ausführlich erläutert, wobei es zu den letzten beiden Themen auch einen praktischen Teil gibt.

Für Thema 1 wird die historische Entwicklung im Bereich der maschinellen Entschlüsselung von Sicherheitscodes betrachtet. Dabei wird auf den ersten Meilenstein dieses Gebietes, die Enigma dechiffriere Maschine von Alan Turing geschaut und betrachtet, wie sich die Komplexität von Entschlüsselungsverfahren bis heute verändert hat.

Im Thema gesellschaftliche Auswirkungen von \ac{AI} wird auf die Frage eingegangen, welche Jobs und Tätigkeiten künftig von künstlicher Intelligenz übernommen werden können und in welchem Zeitlichen Korridor dies geschehen kann. Damit verbunden wird jedoch auch aufgezeigt, wie viele neue Tätigkeitsbereiche \ac{AI} der Menschheit eröffnet hat.

Der dritte Themenbereich vergleicht zwei Vorgehensweisen für das automatisierte Handeln mit Krypto Währungen. Dafür wird ein Baselinemodell Anhand eines \enquote{buy low, sell high} Ansatzes implementiert und einem Reinforcement Learning Modell gegenübergestellt. Die beiden Modelle werden dann Anhand von Transaktionen im tatsächlichen Aktienmarkt evaluiert.

Bei den generativen Verfahren der \ac{AI} wird Mithilfe des von Google DeepMind entwickelten WaveNet ein Deep Learning basiertes generatives Modell entwickelt, welches eigenständig Musik produzieren kann.Die Ergebnisse werden dann in einem Podcast über das Projekt evaluiert.

