% !TEX root =  master.tex
\chapter{Auswirkungen von \ac{AI} auf den Arbeitsmarkt}

Künstliche Intelligenzen können um weites besser zeichnen, Bücher schreiben oder auch Schach und Go spielen als Menschen. \cite{jerzy_2019} Computersysteme können aber nicht nur als Agent zum Spielen von Computerspielen genutzt werden, sondern sind in vielen Bereichen auch eine Unterstützung und Entlastung für den Menschen im Alltag. So gibt es bereits intelligente Lösungen in der Logistik, zur Planung von Lagern und den dort arbeitenden Maschinen oder aber auch selbstfahrende Fahrzeuge in den unterschiedlichsten wirtschaftlichen Bereichen. Dabei seien als Beispiel autonome Busse auf Messegeländen oder eigenständige Trailer zum Transport von Containern in Großhäfen genannt. Das Potential dieser Systeme ist dabei fast grenzenlos. \cite{bachmann_2020}

Bei der Nennung dieser Beispiele fällt auf, dass alle Bereiche, die von \ac{AI} unterstützt werden, keine grundsätzlich neuen Erfindungen sind. Container wurden vorher von Menschen mit Lastkraftwagen an die richtige Position gefahren. Diese Mitarbeiter werden nun durch Maschinen ersetzt. Um genau diese Thematik geht es im folgenden Artikel. Es wird die Frage gestellt, welche Arbeitsplätze durch den Einsatz von \ac{AI} bereits zum Teil weg gefallen sind, welche Bereiche noch bevorstehen, aber auch neue Jobs, die durch stärkere Nutzung von künstlicher Intelligenz entstanden sind, fließen in die Betrachtung ein.

Das Magazin Capital hat im Jahr 2019 ein Ranking von Berufen veröffentlicht, welche in Zukunft von \ac{AI} Systemen übernommen werden können. Am Ende der Liste stehen die Jobs von Verkäufer*innen sowie Service- und Pflegekräften. \cite{jerzy_2019} In einigen Geschäften finden Kunden bereits Roboter, welche Ihnen bei der Produktberatung zur Seite stehen. Als bestes Beispiel sei der Care-O-bot des Fraunhofer Instituts genannt, welcher neben dem Verkauf von Elektronik auch älteren Menschen den Alltag erleichtert und Pflegekräfte unterstützt. \cite{jerzy_2019}
An der Spitze der Liste finden sich Berufe, wie Börsenhändler*innen, Journalist*innen und Busfahrer*innen. \cite{jerzy_2019} Vor allem im Börsenhandel ist der Trend zu künstlichen Intelligenzen stark zu spüren. Mit dem automatisierten Handel ist es möglich binnen von Sekunden auf das Marktgeschehen zu reagieren und entsprechen zu handeln. \cite{jerzy_2019} Laut den Aussagen der Capital, Zitat: \enquote{soll der Anteil des sogenannten Algo-Tradings bei etwa 60 Prozent liegen}. \cite{jerzy_2019} Eine Studie der Universität Oxford geht indes davon aus, dass Roboter nahezu 50 Prozent der Jobs in den USA binnen 20 Jahren übernehmen werden. \cite{bachmann_2020} Die Wirtschaftswoche berichtet von 35 Prozent bis in die frühen 2030er Jahre auf dem deutschen Arbeitsmarkt, sich auf eine aktuelle Studie des Beratungsunternehmens PwC beziehend. \cite{guldner_2017}
Das Magazin Rocket Zeigt auch die positiven Effekte von unterstützenden Systemen auf. So sind \ac{AI} unterstützte Systeme in der Lage Krebs oder andere Krankheiten schneller und genauer erkennen zu können als Ärzte.  \cite{bachmann_2020} Bachmann führt dazu weiter aus, Zitat: \enquote{Schon jetzt existieren Softwares, die in der Lage sind, Steuererklärungen oder Versicherungsanträge vollautomatisch zu prüfen}. \cite{bachmann_2020} Das zeigt vor allem auf die sehr diversen Einsatzbereiche von künstlicher Intelligenz, welche vom ausführenden Gewerbe über Dienstleitungen bis hin zu reinen Verwaltungsaufgaben reichen.

Auf der einen Seite sind viele Berufsfelder vom Stellenabbau durch den Einsatz von \ac{AI} bedroht, Experten zufolge wird die zunehmende Digitalisierung jedoch auch viele neue Arbeitsplätze schaffen. \cite{bachmann_2020} Nach einer Europäischen Studie sind im vergangenen Jahrzehnt 1,6 Millionen Arbeitsplätze durch die Nutzung Intelligenter Systeme entfallen, jedoch sind dafür im selben Zeitraum mehr als doppelt so viele neue Stellen entstanden. \cite{bachmann_2020} Bachmann führt dazu aus, Zitat: \enquote{Schließlich sind es Menschen in den entsprechenden IT Jobs, die die Künstlichen Intelligenzen entwickeln, trainieren und sich um ihre Wartung kümmern}. \cite{bachmann_2020} Das zeigt zum einen darauf, dass komplett neue Berufsbilder entstehen, aber auch, dass immer noch Menschen benötigt werden, welche diese Systeme einrichten und in ihrer Ausführung auch überwachen und kontrollieren.

In diesem Artikel wurden als Beispiele überwiegend ausführende Berufe genannt. Guldner spricht jedoch auch davon, dass, Zitat: \enquote{die so genannte “white collar automation”, also das computer-bedingte Wegrationalisieren von Bürojobs, kein Hirngespinst ist} und Belegt seine Aussage mit Einschätzungen von Mangern aus Dax-Konzernen. \cite{guldner_2017}