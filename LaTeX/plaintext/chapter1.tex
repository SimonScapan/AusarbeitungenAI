% !TEX root =  master.tex
\chapter{Historische Entwicklung der Kryptographie}

Mit der Entwicklung von Computern und der damit einhergehenden Verarbeitung von Daten darf die Frage nach dem Datenschutz und der Datensicherheit nicht außer Acht gelassen werden. Analoge Daten in Form von Papier können in Tresoren und verschlossenen Koffern transportiert werden. Dieses System wurde bereits bei den Anfängen der digitalen Datenübertragung beachtet und es wurden bestimmte Sicherheiten eingebaut.
Das aufbrechen dieser Verschlüsselungsmethoden ist dem Gebiet der Kryptographie zuzuordnen. In diesem Fachbereich werden Methoden entwickelt um komplexe Verschlüsselungen aufzubrechen.
Die Sicherheit von Daten, die durch verteilte Systeme fließen, hat in der heutigen Gesellschaft einen hohen Wert. Dennoch ist es notwendig diese Verschlüsselungen aufbrechen zu können, um zum Beispiel bei Kriminellen Vorgängen diese frühzeitig zu unterbinden.

Im weiteren Verlauf werden einige wichtige Meilensteine der Kryptographie näher dargestellt. Dabei liegt das Augenmerk auf immer komplexeren Verschlüsselungen und den damit einhergehenden komplizierteren Anwendungen, zum entschlüsseln eben dieser Codes. Die Historische Entwicklung wird dabei auf zwei Wichtige Ereignisse der Vergangenheit konzentriert und gibt weiterhin einen Blick in die Zukunft in Sachen Quantenkryptographie. Die Folgenden Meilensteine werden näher betrachtet:
\begin{itemize}
	\item Turing - Bombe
	\item VENONA - Projekt
	\item Quantenkryptographie
\end{itemize}

Dabei ist zu bemerken, dass in den ersten beiden Systematiken keine künstlichen Intelligenzen eingesetzt wurden, sondern eher regelbasierte Methoden. Jedoch sind diese Überlegungen die Grundlage für die heutige Forschung auf dem Gebiet der Kryptographie.

\subsubsection{Turing - Bombe}

Im ersten Weltkrieg war es für die Kriegsparteien von höchster Bedeutung, dass die Kommunikation zwischen den einzelnen Organen des Militärs verschlüsselt ablief. Im Falle einer nicht verschlüsselten Kommunikation könnte der Feind mithören und für das Kriegsgeschehen relevante Informationen abgreifen.
Diese Verschlüsselung der Kommunikation wurde auf deutscher Seite mittels einer Chiffriere Maschine, der Enigma realisiert. Arthur Scherbius patentierte diese Verschlüsselungsmaschine bereits 1918 für die Anwendung im militärischen aber auch zivilen Umfeld. \cite{bloemer_2012} Die Enigma wurde jedoch erst ab 1930 von der deutschen Reichswehr eingesetzt. \cite{bloemer_2012}
Die äußere Ansicht der Enigma ähnelt stark der einer Schreibmaschine. Neben der 26 Buchstaben umfassenden Tastatur besaß die Enigma ein Lampenfeld mit je einer Glühbirne für die den entsprechenden Buchstaben auf der Tastatur. \cite{bloemer_2012} Das Steckerbrett und der Walzensatz im inneren der Enigma bilden dabei den kryptographischen Kern. \cite{bloemer_2012}
Nach Oepen und Höfer kann die Geschichte zur Entschlüsselung der Enigma in zwei Phasen aufgeteilt werden, so sind, Zitat: \enquote{vor dem Krieg vor allem die Leistungen der Polen hervorzuheben. Nach der Besetzung Polens durch die Deutschen wurde die Arbeit der polnischen Kryptoanalytiker von den Briten fortgesetzt.}
Dies Geschah mit der deutschen Invasion auf Polen, wodurch die Polen beschlossen ihre gesamten Forschungsergebnisse an die Alliierten Einheiten weiterzureichen. \cite{oepen_hoefer_2007}

Die Nutzung der Enigma lief wie folgt ab:
\enquote{Durch Drücken eines Buchstabens auf der Tastatur wurde ein Stromkreis geschlossen. Dieses führte zum Aufleuchten eines Buchstabens im Lampenfeld, der die Verschlüsselung des auf der Tastatur gedrückten Buchstabens war. Eine Nachricht wurde verschlüsselt, indem sukzessive die Buchstaben der Nachricht auf der Tastatur gedrückt und die danach im Lampenfeld aufleuchtenden Buchstaben notiert wurden. Die so erhaltene verschlüsselte oder chiffrierte Nachricht, der Chiffretext, wurde per Funk übertragen.} \cite{bloemer_2012}

Die Funktionsweise der Enigma ist dagegen um weites Komplexer und wird deshalb außer Acht gelassen, da es den Rahmen dieser Ausarbeitung sonst überschreiten würde.

Durch die Konstellation der Enigma mit den drei inneren Walzen ist es möglich, nacheinander 26 · 26 · 25 = 16 900 unterschiedliche Substitutionen zu generieren. \cite{bloemer_2012} Erst nach 16 900 Buchstaben erreicht die Enigma ihren ursprünglichen Zustand und berechnet so eine  polyalphabetische Substitutions Chiffre mit einer Blocklänge von 16 900. \cite{bloemer_2012} Durch die Enigma verschlüsselte Nachrichten waren in den meisten Fällen jedoch kürzer als 16 900 Buchstaben, was den deutschen Geheimdiensten eine fälschlicherweise hohe Sicherheit eben diesen Verfahrens suggerierte. \cite{bloemer_2012}

Das deutsche Militär war im gesamten Krieg der festen Überzeugung dass die Enigma sicher sei. \cite{bloemer_2012} Durch die gezielte Kriegsführung der Alliierten wussten die deutschen um Lecks in der Kommunikation, schoben dies jedoch auf gute Spione und stellten die Sicherheit der Enigma nie gänzlich in Frage. \cite{bloemer_2012}

Es gelang den Alliierten einen Großteil deutscher Nachrichten abzufangen, die Gründe dafür nach Bloemer sind: \cite{bloemer_2012}
\begin{itemize}
	\item Speziell konstruierte Maschinen, die Turing-Welchman-Bombe, welche einen Großteil des Codebrechens automatisierten
	\item Entwurfsfehler der Enigma
	\item Mangel in Bedienung und Nutzung, vor allem schlechtes Schlüsselmanagement
	\item (Kriegs-)Glück, zum Beispiel Aufbringen von U-Booten mit unversehrten Enigma-Maschinen und Schlüsselbüchern
\end{itemize}

Die Systematik hinter den Entschlüsselungen, welche von Turing und seinem Team im Bletchley Park benutzt wurden sind recht einfach zu erklären. Die von den deutschen verschlüsselten Nachrichten beinhalten immer wiederkehrende Textfragmente, welche sich jeden Tag oder auch jede Nachricht glichen. Beispiele dafür sind der Wetterbericht aber auch der nationalsozialistische Gruß \enquote{Heil Hitler}. \cite{bloemer_2012} Diese Fragmente, im Bletchley Park genannten Cribs, waren Indizien für die schnelle Entschlüsselung der Enigma. \cite{bloemer_2012} Bloemer beschreibt den Vorgang, Zitat: \enquote{Kannte man ein Crib und seine genaue Position in einer Nachricht, versuchte man, diejenigen Schlüssel zu bestimmen, die aus dem Crib den entsprechenden Teil des Chiffretextes erzeugten. Unter den (hoffentlich) wenigen gefundenen Kandidaten wurden schließlich, durch Probieren, Teile des Tagesschlüssels, insbesondere die Walzenlage, ermittelt.} \cite{bloemer_2012}

Dieses Prinzip der Entschlüsselung machte sich Turing kurze Zeit später zu Nutze und entwickelte eine Maschine, welche die einzelnen Kombinationen durchgeht und auf Plausibilität prüft. Die Turing-Welchman-Bombe war geboren. Ab Mai 1941 konnte der sich täglich ändernde Schlüssel der 3-Walzen-Enigma innerhalb weniger Stunden gebrochen werden. \cite{bloemer_2012}

Das Prinzip der Verschlüsselung war mit dem Vertauschen von Buchstaben auf den ersten Blick recht simpel, durch die Komplexität der Enigma jedoch trotzdem eine Mammutaufgabe für Welchman und Turing. Mit ihrer Entschlüsselung haben sie maßgeblich dem Kriegsgeschehen beigetragen und gingen damit in die Geschichte ein.


\subsubsection{VENONA - Projekt}

Auch im kalten Krieg zwischen der Sowjet Union und den Alliierten war die Notwendigkeit der Entschlüsselung von Nachrichten ein hoch relevantes Thema. Der \ac{SIS} der US-Armee nahm im Februar 1943 seine Arbeit als Vorläufer der \ac{NSA} auf und leitete das top-geheime Programm VENONA ein. \cite{venona_NSA}
Nach dem zweiten Weltkrieg kam es zu einem Kommunikationsstau zwischen Moskau und einer Auslandsvertretung der Sowjet Union, wodurch es den Alliierten gelang über 200.000 Nachrichten abzufangen. \cite{simkin_2020} Ein Team von Kryptoanalytikern bekam die Aufgabe eben diese zu entschlüsseln. Das Projekt VENONA wurde geboren und die USA konnte so mehrere Tausende Sowjetische Nachrichten dechiffrieren. \cite{simkin_2020} Die Aufgabe des Programms bestand darin, die diplomatische Kommunikation der Sowjetunion zu untersuchen und auszunutzen. \cite{venona_NSA}
Obwohl es fast zwei Jahre dauerte, bis amerikanische Kryptologen die KGB-Verschlüsselung knacken konnten, lieferten die durch diese Transaktionen gewonnenen Informationen der US-Führung einen Einblick in die sowjetischen Absichten, bis das Programm 1980 eingestellt wurde. \cite{venona_NSA}
Da es sich um ein Geheimes Projekt der \ac{NSA} handelte sind nähere Informationen zur Art der Entschlüsselung nicht bekannt. Da es sich jedoch um die Nachkriegszeit handelt kann davon ausgegangen werden, dass hierfür komplexe computergestützte Systeme zum Einsatz kamen.


\subsubsection{Quantenkryptographie}

Bei den zuvor genannten Brute-Force Heuristiken zum Dechiffrieren von Verschlüsselungen, was bedeutet, dass unterschiedliche Schlüssel so lange probiert werden, bis der richtige gefunden wurde, ist es nicht möglich von vorn herein zu wissen, dass eine nicht autorisierte Einheit eine Nachricht abfängt. Anders ist es bei der Quantenkryptographie. 
Hier werden die Informationen nicht als elektromagnetische Impulse, sondern als polarisierte Photonen transportiert. Das hat laut den Ergebnissen von Wootters und Zurek den Vorteil, dass Informationen nicht störungsfrei kopiert werden können, was einen unbemerkten Versuch des Abhörens unmöglich macht. \cite{rass_schartner_2002}
Jeder Abhörversuch führt somit unweigerlich zu einem unnatürlichen Anstieg der Fehlerrate an beiden Enden des Kommunikationsweges, welches bei Erkennung zu einerm sofortigen Stoppen der Kommunikation führt. \cite{rass_schartner_2002}
Die in der Quantenkryptographie genutzten Photonen sind relativ einfach zu erzeugen und können Mithilfe von Glasfaserkabeln sehr schnell transportiert werden. \cite{tittel_brendel_gisin_ribordy_zbinden_1999} Diese Technologie wurde in den letzten Jahren auch hierzulande stark ausgebaut, was eine Grundlage für sichere Kommunikation bietet.
Mit Kommunikationsreichweiten von 20 bis 30 Kilometern und Quantenbit-Fehlerraten im niedrigen Prozentbereich ist es eingach Abhörangriffe aufzuspüren und die sichere Übertragung einer Nachricht gewährleisten zu können. \cite{tittel_brendel_gisin_ribordy_zbinden_1999}
Nach Rass und Schartner gilt die, Zitat: \enquote{Quantenkryptographie als Schlüsseltechnologie der kommenden Jahrzehnte. Über 20 Jahre hat die Evolution von der Idee bis zu den ersten Prototypen gedauert, die bereits heute demonstrieren, dass zukünftige Netzwerke mit hoher Sicherheit optisch sein werden.} \cite{rass_schartner_2002}

\subsubsection{Fazit}
Die Dechiffriere Maschinen von Alan Turing und die im VENONA Projekt sind per Definition keine künstlichen Intelligencen, da sie lediglich Systematiken verfolgen, jedoch bilden sie die Grundlage für die Heutige Forschung im Bereich der Kryptographie. So konnten Forscher Mithilfe einer künstlichen Intelligenz das Voynich-Manuskript, auch bekannt als das Buch, das keiner Lesen kann, entschlüsseln. \cite{business_insider_2018}
Das heißt, wir sind heute in der Lage komplexe Verschlüsselungen zu Lösen, da anhand von \ac{AI} Wörter oder Sätze erkannt werden können, auch wenn nur ein Bruchteil von Informationen gegeben ist. Dies bedeutet aber nicht, dass wir in einer gläsernen Welt leben, in der alles durch eine \ac{AI} entschlüsselt werden kann.
Lediglich Wörtliche Chiffren sind so erkennbar. Digitale Verschlüsselungen wie der \ac{AES} sind immer noch die sichersten Methoden der verschlüsselung Digitaler Daten.