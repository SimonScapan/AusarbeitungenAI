% !TEX root =  master.tex
\chapter{Zusammenfassung der Forschungsthemen}
\nocite{*}

Die folgende Aufzählung erläutert in Kurzform, warum die einzelnen Lernziele durch die hiermit vorliegenden Ausarbeitungen erreicht wurden:

\begin{description}
	\item [Historische Entwicklung der \ac{AI}]\hfill \\
	Im ersten Kapitel wurden die unterschiedlichen Herausforderungen der Kryptographie in den verschiedenen Zeitlichen Abschnitten der Geschichte betrachtet. Dabei fällt zum einen auf, dass es gesellschaftlich wichtig ist die Möglichkeit zu haben, um Verschlüsselungen aufzubrechen, jedoch sollte dies auch nur im Notfall und von autorisierten Organen durchgeführt werden. Heute sind digitale Nachrichten mit weit komplexeren Methoden verschlüsselt als zu Zeiten des zweiten Weltkriegs. Aber bereits damals hat es Jahre gedauert, um die Enigma maschinell zu dechiffrieren. Digitale Sicherheit ist ein stets wichtiger werdendes Thema auch in der heutigen Zeit. Die historische Entwicklung aber zeigt, dass der Krieg sicher anders ausgehen würde, könnten die Alliierten deutsche Nachrichten nicht lange unbemerkt abfangen.
	Das Aufzeigen dieser unterschiedlichen Aspekte und Blickweisen auf die Sicherheit von Verschlüsselungen sind Indizien auf erfolgreiches Abschließen dieses Lernabschnittes.
	
	\item [Gesellschaftliche Auswirkungen]\hfill \\
	Dieser Lernabschnitt wurde erreicht, indem eine immer aktueller werdende und gesellschaftlich wichtige Frage aufgegriffen wurde: \enquote{Wird \ac{AI} unterschiedliche Berufe ersetzen?}. Hierbei wurde beleuchtet, welche Jobs bereits heute von \ac{AI} unterstützt werden und wo diese Unterstützung so weit fortgeschritten ist, dass der Mensch nur noch die Maschine kontrolliert. Des Weiteren wurden Berufe aufgezählt, welche in Zukunft von \ac{AI} unterstützt oder gar komplett übernommen werden könnten.
	
	\item [Reinforcement Leaning Methode]\hfill \\
	Für diesen Themenabschnitt wurden zwei Algorithmen zum automatisierten Handeln an Börsen entwickelt. Zusätzlich dazu wurden die Ergebnisse in einem wissenschaftlichen Paper festgehalten und evaluiert. Das Lernziel ist mit zwei Einreichungen als abgeschlossen zu bewerten.
	
	\item [Generatives Verfahren] \hfill \\
	Das letzte Lernziel wurde erreicht indem ein generatives Verfahren zur Produktion von Musik während der Vorlesung erläutert wurde. Hierbei handelt es sich im Detail um das von DeepMind veröffentlichte WaveNet, welches ermöglicht, anhand von Audiodateien neue Musik zu generieren. Dabei wurde auf das Modell des WaveNet eingegangen aber auch auf Probleme und Chancen gezeigt.
\end{description}